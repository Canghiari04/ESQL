\documentclass{article}
\usepackage[utf8]{inputenc}
\usepackage[T1]{fontenc}
\usepackage{lipsum}
\usepackage{graphicx}
\usepackage{amsmath}
\usepackage[margin=1in]{geometry}
\usepackage{titlesec}
\usepackage{enumitem}
\usepackage{geometry}
\usepackage{tabularx}
\usepackage{caption}
\usepackage{fixltx2e}
\usepackage{booktabs}
\usepackage{float}  

\titleformat{\section}
{\LARGE\bfseries}{\thesection}{1em}{}

\titleformat{\subsection}
{\Large\bfseries}{\thesection}{1em}{}

\titleformat{\subsubsection}
{\normalfont\large\bfseries}{\thesection}{2em}{}

\begin{document}

\setcounter{subsection}{0}

\pagestyle{empty}

\newgeometry{left=2cm, right=2cm}
\begin{titlepage}
\begin{center}
    {{\Large{\textsc{Alma Mater Studiorum - Università di Bologna}}}}
    \rule[0.1cm]{\textwidth}{0.1px}
    \rule[0.5cm]{\textwidth}{0.6px}\\
    {\fontsize{12}{13}{SCUOLA DI SCIENZE \\ Corso di Laurea in Informatica per il Management}}
\end{center}

\vspace{50px}

\begin{center}
    {\LARGE{{\bf Piattaforma ESQL}}}\\
\end{center}

\vspace{115px}
\par
\noindent
\begin{minipage}[t]{0.04\textwidth}
~
\end{minipage}
\begin{minipage}[t]{0.4\textwidth}
\end{minipage}
\hfill
\begin{minipage}[t]{0.4\textwidth}\raggedleft
{\fontsize{12}{13}{SVOLTA DA:}\\
\fontsize{12}{13}{\it Canghiari Matteo \\ De Rosa Davide \\ Nadifi Ossama}}
\end{minipage}
\begin{minipage}[t]{0.04\textwidth}
~
\end{minipage}

\vspace*{210px}

\begin{center}
    \large{Anno Accademico 2023/2024}
\end{center}
\end{titlepage}

\subsection{Analisi dei requisiti}
\large
All'interno di questa prima sezione, si adotta un approccio orientato ad un'analisi degli aspetti principali inerenti al progetto, mediante una serie di azioni mirate per rendere il più comprensibile possibile il documento di specifica, attraverso la scelta del corretto livello di astrazione, la standardizzazione della struttura delle frasi oppure tramite la decomposizione del testo in espressioni omogenee.

\subsubsection{Documento di specifica}

\subsubsection{Decomposizione in gruppi di frasi}
Di seguito sono descritti i concetti essenziali raggruppati sulla base di medesime caratterizzazioni, affinchè sia definito un supporto concreto per successive fasi di sviluppo, costituito da:
\begin{itemize}[label={-}]
    \itemsep1px
    \item \textbf{UTENTE} \vspace*{3px}\\ Tutti gli utenti dispongono di: email, nome, cognome e di un possibile recapito telefonico. Gli utenti sono suddivisi in due tipologie: docenti e studenti. 
    \item \textbf{STUDENTE} \vspace*{3px}\\ Gli studenti dispongono di un campo anno di immatricolazione e di un codice alfanumerico. Gli studenti possono svolgere un test, inserendo una o più risposte per ciascun quesito.
    \item \textbf{DOCENTI} \vspace*{3px}\\ I docenti dispongono del nome del dipartimento di afferenza e nome del corso di cui sono titolari. I docenti possono creare delle tabelle di esercizio. Devono essere inseriti dai docenti anche i vincoli di integrità referenziale tra attributi di differenti tabelle di esercizio. In aggiunta ogni docente può creare dei test.
    \item \textbf{TABELLE\_ESERCIZIO} \vspace*{3px}\\ Ogni tabella di esercizio dispone di nome, data di creazione, un campo num\_righe. Inoltre, ogni tabella di esercizio dispone di un insieme di attributi.
    \item \textbf{ATTRIBUTO} \vspace*{3px}\\ Ogni attributo dispone di un nome, un tipo e può essere parte della chiave primaria della tabella di esercizio. 
    \item \textbf{TEST} \vspace*{3px}\\ Ogni test dispone di un titolo univoco, una data di creazione e di una possibile foto. Ogni test include una serie di quesiti. Ogni test dispone di un campo VisualizzaRisposte, se settato a True, le risposte dei quesiti diventano visibili agli studenti, altrimenti rimaranno nascoste.
    \item \textbf{QUESITO} \vspace*{3px}\\ Ogni quesito dispone di un numero progressivo, ma solo all'interno della relazione che lo contraddistingue con l'entità test, un livello di difficoltà, un campo descrizione, un campo num\_risposte e si riferisce ad una o più tabelle di esercizio. I quesiti sono esclusivamente di due categorie: domande a risposta chiusa oppure quesiti di codice.
    \item \textbf{DOMANDA\_CHIUSA} \vspace*{3px}\\ La domanda chiusa dispone di una serie di opzioni di risposta. Nel caso di quesiti a domanda chiusa, la risposta consiste in una dell'opzioni disponibili. 
    \item \textbf{OPZIONI\_RISPOSTA} \vspace*{3px}\\ Ogni opzione dispone di una numerazione, univoca rispetto ad uno specifico quesito, ed un campo di testo. 
    \item \textbf{CODICE} \vspace*{3px}\\ Il quesito di codice dispone di una o più soluzioni. Nel caso di quesiti di codice, la risposta consiste in un campo di testo.
    \item \textbf{SKETCH\_CODICE} \vspace*{3px}\\ Gli sketch di codice in SQL implementano query che restituiscano quanto richiesto dal quesito.
    \item \textbf{COMPLETAMENTO} \vspace*{3px}\\ Si vuole tenere traccia del completamento del test, ossia: data di inserimento della prima risposta, data di inserimento dell'ultima risposta, stato.
    \item \textbf{RISPOSTA} \vspace*{3px}\\ Ogni risposta dispone di un campo di esito, che può valere True o False a seconda che la risposta fornita dallo studente coincida con l'opzione del quesito a domanda chiusa oppure che la risposta produca l'output desiderato nel caso di quesiti di codice.
    \item \textbf{MESSAGGI} \vspace*{3px}\\ Ogni messaggio dispone di un titolo, un campo testo, una data di inserimento, e fa riferimento ad uno specifico test. Il messaggio può essere inviato da un docente oppure da uno studente, nel primo caso i destinatari saranno tutti gli studenti mentre nel secondo caso il destinatario sarà un determinato docente.
\end{itemize}

\subsubsection{Lista delle operazioni}
\dots

\subsubsection{Tavola media dei volumi}
\dots

\subsubsection{Glossario dei dati}
\dots

\subsection{Progettazione concettuale}
\large
\subsubsection*{Modello E-R}
\begin{figure}[H]
    \includegraphics[width=1\textwidth]{foto1.png}
    \caption{Modello E-R precedente alla raffinazione.}
\end{figure}

\subsubsection*{Dizionario delle entità}
\begin{table}[H]
    \centering
    \begin{tabularx}{\textwidth}{|X|p{5cm}|p{5cm}|X|}
        \hline
        Entità & Descrizione & Attributi & Identificatore \\
        \hline
        Utente & Utilizzatore dell'applicativo & Email, Nome, Cognome, Telefono & Email \\        
        \hline
        Docenti & Docente creatore e ideatore di quesiti e tabelle di esercizio & Nome\_Dipartimento, Nome\_Corso & Email \\
        \hline
        Studente & Alunno che interagisce con l'applicativo per la risoluzione dei quesiti posti & Anno\_Ipxatricolazione, Codice & Email \\
        \hline
        Tabelle\_Esercizio & Tabelle che costituiscono parte dei meta-dati per la realizzazione di test & ID, Nome, Data\_Creazione, Num\_Righe & ID \\
        \hline
        Attributi & Attributi i quali costituiscono la seconda entità essenziale per i meta-dati, legati alla realizzazione di test da sopxinistrare & ID, Tipo Nome, Chiave\_Primaria & ID \\
        \hline
        Test & Test indica l'insieme di quesiti svolti dagli studenti e creati dal docente & Titolo, Foto, Data\_Creazione, Visualizza\_Risposte & Titolo \\
        \hline
        Quesito & Domanda relativa a tematiche svolte nel corso & Numero, Difficoltà, Descrizione, Num\_Risposte & Numero \\
        \hline
        Domanda\_Chiusa & Tipologia di Quesito, rappresentante una domanda a scelta multipla & . & Numero \\
        \hline
        Sketch\_Codice & Tipologia di Quesito, richiedente la formulazione di query SQL & . & Numero \\
        \hline
        Messaggi & Comunicazioni ricevute e inviate tra docenti e studenti & ID, Titolo, Data\_Inserimento & ID \\
        \hline
    \end{tabularx}
    \caption{Descrizione delle entità del modello E-R precedente alla raffinazione.}
\end{table}

\subsubsection*{Dizionario delle relazioni}
\begin{table}[H]
    \centering
    \begin{tabularx}{\textwidth}{|X|p{5cm}|p{3cm}|X|}
        \hline
        Relazione & Descrizione & Componenti & Attributi \\
        \hline
        Creazione & Creazione da parte di docenti di tabelle di esercizio e quesiti & Docenti, Tabelle\_Esercizio, Test & . \\
        \hline
        Completamento & Completamento di un test sopxinistrato da parte degli studenti & Studente, Test & Stato, Data\_UltimaRisposta, Data\_PrimaRisposta \\
        \hline
        Invio & Invio di messaggi da parte di docenti e studenti & Studente, Docenti, Messaggi & . \\
        \hline 
        Pubblicazione & Pubblicazione di comunicazioni afferenti ad uno specifico test & Messaggi, Test & . \\
        \hline
        Ricezione & Ricezione di messaggi emessi da studenti oppure da docenti & Messaggi, Docenti & . \\
        \hline
        Risposta & Risposta formulata dagli studenti in relazione ad uno specifico quesito & Studente, Quesito & Esito \\
        \hline
        Composizione & Composizione di un insieme di quesiti rispetto ad un determinato test & Quesito, Test & . \\
        \hline
        Afferenza & Afferenza dei quesiti ideati relativamente a tabelle di esercizio & Quesito, Tabelle\_Esercizio & . \\
        \hline
        Combinazione & Combinazione di attributi per la costruzione di tabelle di esercizio & Tabelle\_Esercizio, Attributi & . \\
        \hline
        Soluzione & Soluzione alla query SQL richiesta & Codice, Skecth\_Codice & . \\
        \hline
        Disposizione & Disposizione del numero complessivo di opzioni di risposta relative alla domanda sottoposta & Domanda\_Chiusa, Opzioni\_Risposta & . \\    
        \hline
    \end{tabularx}
    \caption{Descrizione delle relazioni del modello E-R precedente alla raffinazione.}
\end{table}

\subsubsection*{Modello E-R raffinato}
\begin{figure}[H]
    \includegraphics*[width=1.1\textwidth]{foto2.png}
    \caption{Modello E-R successivo alla raffinazione.}
\end{figure}

\subsection{Progettazione logica}
\large
\dots 

\subsection{Normalizzazione}
\large
\dots 

\subsection{Descrizione delle funzionalità}
\large
\dots 

\subsection{Codice SQL}
\large
\dots 


\subsubsection*{Costo operazionale}


\end{document}